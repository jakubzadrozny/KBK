\documentclass[polish]{kbk}
\setlength{\parskip}{0.4em}
\usepackage{url}

\begin{document}

\author{Jakub Zadrożny}
\title{Jak korporacje wykorzystują media społecznościowe?}

\maketitle

\section{Wstęp}
Pojawienie się mediów społecznościowych z pewnością zmieniło sposób w jaki ludzie korzystają z internetu. Z roku na rok gromadzą one coraz więcej użytkowników - we wrześniu 2016 roku już blisko 60\% internautów deklarowało używanie któregoś z nich \cite{stats}. Poziom integracji portali społecznościowych z naszym życiem oraz sposób, w jaki z nich korzystamy, czyni je olbrzymią bazą danych, zawierającą informacje nie tylko o ich użytkownikach, ale także o ich znajomych, a nawet o jakości relacji między nimi. Te informacje okazały się niezwykle cenne dla świata biznesu - poznając nasze prywatne życie, korporacje są w stanie dopasować do nas swoją ofertę oraz podjąć istotne decyzje na nasz temat. Wraz z rozwojem technologii uczenia maszynowego przetwarzanie tej bazy stało się możliwe na niespotykaną dotąd skalę. Cały proces mógł wreszcie zostać zautomatyzowany, a analizy jednej osoby można obecnie dokonać za pomocą kilku kliknięć w kilka sekund.

Już kilka lat temu pojawiły się pierwsze komercyjne zastosowania analizy mediów społecznościowych, a co roku na rynek wchodzą kolejne. Wydaje się, że ta metoda dobrze sprawdza się w korporacjach i w przyszłości możemy obserwować jej dalszy rozwój. Jednak czy dostęp korporacji do naszych prywatnych informacji nie pozwala im uzyskać nad nami zbyt dużej kontroli? I czy w obliczu takich działań komercyjnych nasza prywatność nie jest dodatkowo zagrożona?

\section{Zdolność kredytowa}
Właściwa ocena ryzyka kredytowego jest niezbędna do prawidłowego działania instytucji finansowych. W krajach rozwiniętych korzysta się obecnie ze skomplikowanych metod statystycznych wykorzystujących m.in. historię kredytową danego klienta. Problem pojawia się, gdy mamy zbyt mało danych, by dobrze ocenić zdolność kredytową (np. brak historii) danej osoby. Ta komplikacja jest szczególnie dotkliwa na rynkach wschodzących, gdzie o ogromnej liczbie ludzi nie ma praktycznie żadnych danych kredytowych, a instrumenty finansowe są z tego powodu ciężko dostępne. 

Wraz z początkiem mediów społecznościowych i rozwojem analizy \textit{Big Data} narodził się pomysł na rozwiązanie problemu rynków wschodzących - automatyczna ocena zdolności kredytowej na podstawie aktywności wnioskującego w sieciach społecznościowych. Takie rozwiązanie posiada kilka istotnych zalet - pozwala zmniejszyć koszty i czas podejmowania decyzji oraz umożliwia ocenę zdolności osób, o których posiadamy niewiele danych. Ostatecznie może ono również działać równolegle do standardowej analizy, co powinno zwiększyć efektywność. Tylko czy rzeczywiście to, co publikujemy na Facebooku mówi o nas tak wiele, by stwierdzić, czy spłacimy nasz kredyt? Niektóre firmy twierdzą, że znają odpowiedź na to pytanie.

\subsection{Lenddo \cite{cnn-tech, lenddo}}
Jedną z nich jest założone w 2011 roku \textit{Lenddo}. Jeff Stewart, jeden z założycieli, twierdzi, że ludzie sami potrafią zaskakująco skutecznie określić, kto w ich środowisku jest godny zaufania, a od kogo należy trzymać się z daleka. Firma wykorzystuje tą obserwację analizując kontakty na portalach społecznościowych - jeśli wśród naszych znajomych znajdują się osoby, które zalegają ze swoimi ratami, to algorytm obniża naszą wiarygodność. Dodatkowo pod uwagę brana jest również jakość naszej relacji. Im częściej zatem kontaktujemy się z podejrzaną osobą, tym niżej spada nasza własna wiarygodność. Firma stale pozyskuje nowe środki od inwestorów i rozwija swój algorytm - obecnie pod uwagę bierze znacznie więcej czynników niż tylko portale społecznościowe - np. pocztę elektroniczną i dane ze smartfonów.

\subsection{Kreditech \cite{cnn-tech, slate}}
Kolejnym start-upem oferującym błyskawiczną ocenę zdolności kredytowej jest założony w 2012 roku niemiecki \textit{Kreditech}. Analiza przeprowadzana przez tą firmę również wykracza poza same media społecznościowe. Model opracowany przez Kreditech przygląda się również zachowaniu użytkownika w sklepach internetowych (np. Amazon, e-Bay), zawartości urządzenia, z którego wypełniany jest wniosek (system operacyjny, zainstalowane programy), lokalizacji GPS oraz sposobu składania wniosku przez użytkownika. Według firmy, użytkownicy, którzy poświęcają zbyt mało czasu na zapoznanie się z regulaminami, wypełniają formularze wyłącznie wielkimi literami lub podają dane adresowe niezgodne z lokalizacją, mają mniejsze szanse na spłacenie pożyczki w terminie.

\subsection{Kabbage \cite{cnn-tech, kabbage}} 
Rozwiązania stworzone przez Lenddo i Kreditech są skierowane do klientów indywidualnych, jednak istnieją również firmy implementujące podobną technologię w pracy z niewielkimi przedsiębiorstwami. Jedną z nich jest założone w 2009 roku \textit{Kabbage}. Ich rozwiązanie opiera się na podobnych założeniach, jak opisanych powyżej firm, jednak z powodu znacznych różnic pomiędzy klientem indywidualnym, a przedsiębiorstwem, konieczne było zastosowanie nieco innego podejścia. Zgłaszając swój wniosek kredytowy, przedsiębiorca musi udostępnić Kabbage'owi m.in. swoje konta z rejestrami płatności i dostaw do klientów (np. konto PayPal lub eBay). Na podstawie takich rejestrów wylicza się ranking kredytowy przedsiębiorstwa i na jego podstawie podejmuje decyzję.

Wnioskujący ma również możliwość przekazania do analizy kont swojego przedsiębiorstwa w serwisach społecznościowych (np. Facebook, Twitter), co powinno znacząco polepszyć jego ostateczny wynik. Jak twierdzi Kabbage, przedsiębiorstwa, które dołączają swoje konta społecznościowe, o 20\% rzadziej zalegają z płatnościami. Zdaniem Marca Gorlina, współzałożyciela firmy, przedsiębiorstwa, które przywiązują uwagę do obsługi klienta na Facebooku, czy Twitterze, prawdopodobnie lepiej radzą sobie również w innych segmentach swojego biznesu, np. magazynowaniu, czy dostawie.

\subsection{Telefonia komórkowa \cite{slate, cignifi}}
Zdaniem niektórych firm nawet dostęp do portali społecznościowych nie jest niezbędny do oceny ryzyka kredytowego. Wystarczyć ma jedynie analiza rejestru rozmów telefonicznych. \textit{Safaricom} - największy kenijski operator - analizuje kiedy, gdzie, jak często, jak długo i do kogo dzwonią jej klienci oraz w jaki sposób korzystają z bankowości mobilnej. Gdy nagromadzi wystarczająco dużo danych, by stwierdzić, że klient jest godny zaufania - oferuje mu pożyczkę. W dość podobny sposób działa startup z Cambridge - \textit{Cignifi}. Współpracuje on z operatorami telefonicznymi, by uzyskać dostęp do statystyki użycia sieci przez klienta. \textit{Cignifi} analizuje dane takie jak: miejsca, czas i długość połączeń, numery inicjujące rozmowy, najczęstsze kontakty oraz sposób opłacania rachunków za telefon. Na ich podstawie próbuje następnie odgadnąć jak najwięcej o stylu życia klienta i zarekomendować lub odradzić instytucjom finansowym udzielenie mu pożyczki.

\section{Ubezpieczenia}
Media społecznościowe są dla przemysłu ubezpieczeniowego szczególnie cenne. Właściwie oszacowanie ryzyka jest kluczowe w przypadku wyceny polisy, a aby tego dokonać trzeba jak najgłębiej poznać styl życia prześwietlanej osoby. Nic natomiast nie dostarcza tak wielu bogatych w informacje detali o naszym życiu jak portale społecznościowe. Bardzo często treści, które zamieszczamy, mogą wydawać się nieszkodliwe, jednak w istocie zawierać przesłanki o podwyższonym ryzyku w naszym życiu. Cennymi informacjami dla ubezpieczyciela mogą być np. \cite{insurance-quotes}:
\begin{itemize}
\item niezdrowe nawyki, informacje o stanie zdrowia w przeszłości; mogą to być np. powtarzające się zdjęcia z papierosami;
\item informacje o rodzinie; np. zainteresowanie stronami o tematyce chorób genetycznych;
\item zagrożenia w domu; np. zdjęcia z niebezpiecznymi zwierzętami jak jadowite węże, pająki;
\item niebezpieczne hobby; np. zainteresowanie stronami o tematyce sportów ekstremalnych;
\item zdradzanie zbyt wielu informacji; np. zdjęcia z plaży, na których widać niepilnowany bagaż;
\end{itemize}
Takie informacje istotnie ułatwiają analitykom oszacowanie ryzyka i odpowiednią wycenę polisy. Oczywiście proces prześwietlania mediów społecznościowych każdego klienta z osobna byłby niezwykle kosztowny i czasochłonny, jednak już teraz pojawiają się próby jego komputeryzacji, co pozwoliłoby wdrożyć go na wielką skalę.

\subsection{firstcarquote \cite{guardian, guardian-stop}}
Ciekawe rozwiązanie przedstawił niedawno jeden z największych brytyjskich zakładów ubezpieczeniowych - Admiral. W listopadzie 2016 roku zaprezentował on projekt o nazwie \textit{firstcarquote}, który mając zapewniony dostęp do profilu klienta na Facebooku szukałby cech charakteru, które pozwalają przypuścić, że będzie on dobrym kierowcą. System skierowany był głównie do młodych kierowców kupujących polisy na pierwsze auta, ponieważ to w ich przypadku niemożliwe jest przyznanie zniżki na podstawie dotychczasowej historii. 

Algorytm stara się odgadnąć charakter klienta na podstawie jego zachowania na Facebooku i zmierzyć podobieństwo do wybranych przykładów dobrych kierowców. Według niego sumienność i dobra organizacja to przykłady pożądanych cech. Wśród pozytywnych zachowań znajdują się również: pisanie prostych i przejrzystych zdań, używanie list, umawianie spotkań na konkretne godziny i miejsca (zamiast tylko np. ``dziś wieczorem''). Jedną z cech negatywnych jest z kolei nadmierna pewność siebie - czyli np. nadużywanie wyrażeń ``zawsze'' lub ``nigdy''.

Udział w \textit{firstcarquote} miał być dobrowolny, a w jego efekcie klientom mogła zostać jedynie przyznana zniżka - nigdy podwyższenie ceny. Zdaniem Admirala jego celem było umożliwienie młodym kierowcom uzyskania zniżki szybciej niż po wielu latach budowania historii. Ubezpieczyciel zaznaczał jednocześnie, że pokłada duże nadzieje w tym rozwiązaniu, a projekt był dopiero testem możliwości podobnego systemu. W razie jego powodzenia mógłby zostać wdrożony na szerszą skalę (np. kierowców z dłuższym stażem) oraz z poważniejszymi konsekwencjami - np. podwyższeniem ceny polisy dla niektórych klientów. Ostatecznie rozwiązanie zostało zatrzymane przez samego Facebooka, który uznał, że jego działanie jest niezgodne z polityką platformy, i do dziś nie zostało wdrożone.

\subsection{Odmowa wypłaty \cite{insurance-quotes}}
Informacje znalezione na portalach społecznościowych mogą być również wykorzystane przez ubezpieczycieli do odmowy wypłaty świadczeń z polisy. Z jednej strony pozwala to czasem zidentyfikować i zapobiec próbom wyłudzenia pieniędzy, z drugiej natomiast może dać firmom niesłuszne podstawy do odmowy wypłaty pieniędzy. Z taką praktyką spotkała się Natalie Blanchard pochodząca z Kanady. Natalie przebywała na długoterminowym zwolnieniu lekarskim z powodu zaawansowanej depresji, gdy nagle odcięto jej dopływ pieniędzy z polisy. Gdy skontaktowała się z ubezpieczycielem, by wyjaśnić sprawę, okazało się, że firma uznała, iż jej roszczenia są bezpodstawne. Jako powód zakład podał znalezione na Facebooku zdjęcia z imprezy w barze i podczas opalania na plaży. Natalie tłumaczyła, że był to sposób na ``odskoczenie od problemów'', jednak ubezpieczyciel pozostał przy swoim i zaniechał płatności. Nie chciałbym rozstrzygać, która ze stron była poszkodowana, jednak faktem jest, że treści z portalu społecznościowego pozbawiły Natalie pieniędzy. Nieco mniej kontrowersyjne zdarzenie miało miejsce w 2011 roku w Kalifornii, gdzie trzy kobiety wypożyczyły samochody z dodatkowym ubezpieczeniem, po czym upozorowały wypadek i domagały się odszkodowania. Analizując ich kontakty na Facebooku wywnioskowano, że wszystkie trzy dobrze się znają, chociaż zaprzeczyły temu w oświadczeniach. Ich plan został pogrzebany właśnie przez sprytne użycie mediów społecznościowych przez zakład.

\section{Pracodawcy \cite{social-hiring}}
Wykorzystanie mediów społecznościowych przez korporacje nie zawsze ma bezpośredni związek z rodzajem prowadzonej działalności. Większość firm nie potrzebuje oceniać wiarygodności klienta, ani szacować ryzyka. Jednak każde większe przedsiębiorstwo musi sprawnie zarządzać swoją kadrą, tj. zwalniać nieefektywnych pracowników i zatrudniać nowych. Przy realizacji tego właśnie zadania niektórzy posiłkują się informacjami zdobytymi na portalach społecznościowych.

\subsection{Zwolnienia}
Pracodawcy zależy na rzetelności i efektywności swojego pracownika, dlatego zawsze interesują go informacje, które mogłyby jedną z tych cech podważać. Łatwo dostępnym i bogatym w cenne informacje źródłem są często właśnie profile w mediach społecznościowych. Dlatego niektórzy pracodawcy mogą stale - mniej lub bardziej dokładnie - monitorować aktywność zatrudnionych osób na nich. Nawet jeśli nie robi tego bezpośrednio menadżer, to pewnie wyręcza go jeden ze współpracowników (liczący na potknięcie). Pracodawcę na pewno zainteresowałby fakt, że na portalu społecznościowym ujawniamy poufne informacji, narzekamy na warunki pracy, publikujemy treści w imieniu firmy bez upoważnienia, czy np. szukamy nowego stanowiska. Znanych jest już wiele przypadków zwolnień, które wywołane zostały niepożądanym odkryciem na portalu społecznościowym, a część z nich skończyła się sprawą w sądzie (inicjowaną z różnych stron).

\subsection{Rekrutacja}
Współcześnie media społecznościowe odgrywają również znaczną rolę w procesie rekrutacji nowego pracownika. Rekruterzy wykorzystują je jako szansę na przyjrzenie się kandydatom bliżej niż tylko poprzez CV i listy motywacyjne. Pracodawcy szukają na profilach zarówno dodatkowych powodów by zatrudnić kandydata (np. ciekawe portoflio), jak i przesłanek za jego odrzuceniem. Wśród aktywności, które najbardziej odstraszają rekruterów znajdują się m.in. niewłaściwe lub prowokacyjne zdjęcia, przesłanki o nadużywaniu alkoholu (lub innych używek), rasistowskie i seksistowskie komentarze, złe wyrażanie się o poprzednim pracodawcy lub kiepskie zdolności komunikacji. Menadżerowie do tego stopnia upodobali sobie prześwietlanie portali społecznościowych, że blisko połowę z nich momentalnie zniechęca fakt nieposiadania żadnego profilu publicznego lub ukrywanie jego zawartości poprzez ustawienia prywatności. Znane są nawet przypadki przedsiębiorstw, które wymagają od kandydatów, by posiadali oni konto na portalu społecznościowym.

\section{Kontrowersje}
Firmy pożyczające pieniądze na podstawie analizy niestandardowych danych kredytowych, takie jak Lenddo i Kabbage osiągają spory sukces na rynkach wschodzących - pokaźne inwestycje, wysoka skuteczność modeli (współczynnik Giniego) i stale rosnąca liczba klientów. Lenddo chwali się nawet, że łącząc tradycyjne metody oceny zdolności kredytowej z ich własną, jest w stanie zwiększyć liczbę przyjętych wniosków o 50\%, jednocześnie zmniejszając ryzyko o 12\%. Zdaniem magazynu \textit{Lending Times} firmy takie jak Lenddo są na dobrej drodze do zrewolucjonizowania globalnego systemu oceny zdolności kredytowej, zmuszając organizacje do wzięcia pod uwagę niestandardowych czynników takich jak media społecznościowe. Autorzy porównują nawet przykład Lenddo do Ubera w świecie taksówkarskim, sugerując, że wprowadza on do skostniałej branży powiew świeżości i zmusza ją do modernizacji \cite{lending-times}. W branży ubezpieczeniowej programy takie jak \textit{firstcarquote} dają zakładom możliwość oceny ryzyka w wypadkach, w których standardowa analiza nie rozwiązuje problemu. I chociaż pomysł Admirala został zablokowany, to firmy ubezpieczeniowe już zapowiadają dalszą pracę nad taką formą oceny ryzyka, również z wykorzystaniem tzw. smart-urządzeń (zegarki, opaski fitness, inteligentne domy, itd.) i \textit{Internet of Things} \cite{insurance-further}. Wśród pracodawców już 60\% deklaruje, że przy podejmowaniu decyzji rekrutacyjnej bierze pod uwagę aktywność kandydata na portalach społecznościowych, a odsetek ten z roku na rok wzrasta \cite{social-hiring}.

Wygląda więc na to, że czerpanie ogromu informacji z mediów społecznościowych daje korporacjom znaczne korzyści i powoli staje się już standardem. Oczywiście dla wielu ludzie może to mieć pozytywne konsekwencje, w końcu dzięki tym informacjom firmy umożliwiają wielu osobom zaciągnięcie kredytu, a innym mogą np. zaoferować atrakcyjne zniżki na ubezpieczenie. Okazuje się jednak, że mimo tych pozytywnych aspektów, analizowanie przez korporacje niespotykanej dotąd ilości prywatnych danych nie jest wolne od kontrowersji.

Komputery z roku na rok są coraz mocniejsze, a firmy takie jak Lenddo pozyskują więcej pieniędzy na rozwój swoich algorytmów. Dlatego modele oceny zdolności kredytowej są nieustannie poszerzane o kolejne czynniki i obecnie wykraczają daleko poza analizę jedynie mediów społecznościowych. Dziś analizuje się również połączenia telefoniczne, dane lokalizacji, sposób wypełniania wniosku, zawartość komputera, a nawet komunikację e-mail. Prawdopodobnie do takiej analizy gromadzi się w jednym miejscu większą ilość prywatnych danych niż do jakiegokolwiek innego celu. Każda część informacji analizowanej przez algorytmy pochodzi z reguły od różnych dostawców, którzy nie mają dostępu do informacji między sobą - portale społecznościowe gromadzą swoje dane, dostawcy poczty swoje, a operatorzy komórkowi jeszcze inne. Natomiast do analizy zdolności kredytowej wszystkie te dane łączone są w jedną całość i przetwarzane przez jeden model. Zaopatrzone w tak ogromną ilość naszych prywatnych danych, algorytmy są w stanie poznać nas lepiej niż nasi znajomi, nasza rodzina i być może my sami. Są w stanie nie tylko oszacować naszą zdolność kredytową, ale także np. poznać nasze aktualne potrzeby i dopasować do nich swoje oferty, serwując nam dodatkowo reklamy. Na przykład: algorytm dowie się z naszego Facebooka i poczty, że planujemy zakup nowych mebli, ale chwilowo brakuje nam na nie funduszy; jeśli pozytywnie oceni naszą zdolność kredytową, to automatycznie wyświetli nam reklamy sprzedawców mebli i dodatkowo zaoferuje korzystną pożyczkę, żebyśmy mogli wszystko sfinansować. Posiadając praktycznie nieograniczoną wiedzę o swoich klientach, firmy pożyczkowe mogą do perfekcji opanować sztukę manipulacji i bez problemu skłonić użytkowników do zakupu rzeczy, których oni w rzeczywistości nie potrzebują (szczególnie w dobie konsumpcjonizmu). Taka praktyka mogłaby doprowadzić do uzależnienia ludzi od zaciągania pożyczek. \cite{slate}. 

Warto również zaznaczyć, że ponieważ dostawcy takich algorytmów przetwarzają ogromne ilości naszych danych, to muszą je oni umieć odpowiednio zabezpieczyć. Gdyby nasze dane w jakiś sposób wyciekły z takiego systemu, to atakujący zdobyliby o nas więcej informacji niż przy jakimkolwiek innym włamaniu. Za jednym razem atakującym udałoby się tak naprawdę uzyskać dostęp do naszej poczty, portali społecznościowych, operatora komórkowego i kilku innych źródeł poufnych informacji. Zatem zezwalając korporacjom dostęp, jakiego te obecnie żądają, znacznie zwiększamy ryzyko wycieku naszych danych.

Oczywiście jeśli ktoś ceni sobie swoją prywatność, może nie korzystać z usług takich firm. Jednak dla niektórych ludzi niestandardowa ocena zdolności kredytowej jest jedyną szansą na uzyskanie pożyczki (od której czasami nie da się uciec). Podobna sytuacja może niedługo mieć miejsce w branży ubezpieczeniowej - chociaż \textit{firstcarquote} został stworzony, jedynie by oferować zniżki, to Admiral przyznał, że w razie jego powodzenia rozważy wprowadzenie innych konsekwencji - m.in. podwyższenie ceny polisy dla niektórych klientów. W ciągu ostatnich lat wzrosła również liczba pracodawców, którzy albo niechętnie patrzą na kandydatów niekorzystających z mediów społecznościowych, albo całkowicie ich skreślają. Oznacza to, że z biegiem czasu coraz większa liczba ludzi będzie finansowo zmuszona do sprzedania korporacjom swojej prywatności. Im większy sukces odnoszą takie firmy, tym bardziej przybliża się rzeczywistość, w której na prawo do prywatności musi być nas stać \cite{guardian-stop}.

Nawet jeśli my sami nie ufamy niestandardowej analizie na tyle, by z niej korzystać, nasza prywatność może wciąż być zagrożona. Cała komunikacja i większość interakcji na portalach społecznościowych następuje pomiędzy przynajmniej dwiema osobami - jeśli ktoś jest naszym znajomy, to my jego również. Jeśli więc nasz znajomy udzieli firmie dostępu do swojego konta na profilu społecznościowym dostaną oni informacje nie tylko o nim, ale także o naszej relacji z nim - ile trwa nasza znajomość, jak często się komunikujemy, jak komentujemy i \textit{lajkujemy} swoje posty, itd. Podobnie wygląda nasza sytuacja, jeśli taka firma zażąda od znajomego danych na temat połączeń komórkowych, lub jego poczty elektronicznej - jeśli komunikował się z nami, to wydana zostanie również część naszej korespondencji. Już sam niekontrolowany dostęp do komunikacji z jedną tylko osobą jest niepożądany, a co jeśli wielu naszych znajomych udzieli dostępu tej samej firmie?

Dalsza eskalacja zagarniania zawartości profili w serwisach społecznościowych przez korporacje może doprowadzić do tego, że zmieni się sposób, w jaki ludzie będą takie portale postrzegać. Media społecznościowe staną się głównie zagrożeniem - przecież jeśli zamieścimy tam jakieś kontrowersyjne wpisy, to później możemy spotkać się z odmową udzielenia kredytu, większą ceną polisy lub kłopotem ze znalezieniem pracy. Z drugiej strony pojawi się też szansa na oszukanie systemu poprzez udawanie w mediach społecznościowych osoby, którą chcą w nas widzieć kredytodawcy, ubezpieczyciele i pracodawcy. Wydaje się, że taka perspektywa nie jest zachęcająca dla przedstawicieli serwisów społecznościowych, dlatego też zdarzają się przypadki, w których to same portale ``bronią'' prywatności swoich użytkowników i blokują próby zagarnięcia informacji przez korporacje - np. opisany powyżej projekt \textit{firstcarquote} firmy Admiral zablokowany przez Facebooka z powodu niezgodności z regulaminem platformy. Ten konflikt interesów jest dobrą wiadomością dla internautów - dopóki ludzie cenią sobie swoją prywatność, to w interesie portali społecznościowych będzie leżeć jej ochrona. Oczywiście tak długo, jak długo będzie im się to opłacać \cite{guardian-stop}.

\section{Podsumowanie}
Korporacje już od pewnego czasu sięgają po informacje, które znajdują się w mediach społecznościowych, i wykorzystują je do swoich celów na różne sposoby. Za udostępnienie takich prywatnych informacji internauta jest nierzadko wynagradzany - np. oferuje mu się tańszą pożyczkę lub zniżkę na polisę ubezpieczeniową. Korzyści nigdy nie są jednak dane za darmo - należy za nie zapłacić własną prywatnością. 

Na przestrzeni ostatnich lat coraz więcej firm wykorzystuje i przetwarza informacje, które udostępniamy w mediach społecznościowych. Pracują nad tym nawet korporacje zajmujące się dotąd standardową analizą ryzyka kredytowego w krajach rozwiniętych - takie jak FICO \cite{fico}. Z tego powodu prawo do prywatności w internecie staje się coraz bardziej zagrożone. Na szczęście w interesie mediów społecznościowych jest jego ochrona - przynajmniej tak długo, jak długo ludzie je sobie cenią. Co się natomiast stanie, gdy większość z nas gotowa będzie sprzedać prywatność? Czy będziemy musieli na nią najpierw zarobić? Być może jedynym ratunkiem będą regulacje prawne, a te - jak wiadomo - bywają zawodne.

\begin{thebibliography}{9}

\bibitem{cnn-tech}
  Katie Lobosco.
  \emph{Facebook friends could change your credit score}.
  CNN Tech, 2013.
  \url{http://money.cnn.com/2013/08/26/technology/social/facebook-credit-score/index.html}
  
\bibitem{slate}
  Evgeny Morozov.
  \emph{Your Social Networking Credit Score}.
  Future Tense, 2013.
  \url{http://www.slate.com/articles/technology/future_tense/2013/01/wonga_lenddo_lendup_big_data_and_social_networking_banking.html}
  
\bibitem{guardian}
  Graham Ruddick.
  \emph{Admiral to price car insurance based on Facebook posts}.
  The Guardian, 2016.
  \url{https://www.theguardian.com/technology/2016/nov/02/admiral-to-price-car-insurance-based-on-facebook-posts}
  
\bibitem{guardian-stop}
  Graham Ruddick.
  \emph{Facebook forces Admiral to pull plan to price car insurance based on posts}.
  The Guardian, 2016.
  \url{https://www.theguardian.com/money/2016/nov/02/facebook-admiral-car-insurance-privacy-data}
  
\bibitem{cignifi}
  \emph{How does Cignifi work?}
  Cignifi, 2016.
  \url{http://cignifi.com/creditfinance/}
  
\bibitem{lenddo}
  \emph{Lenddo Score Fact Sheet}.
  Lenddo, 2016.
  \url{https://www.lenddo.com/pdfs/Lenddo-Scoring-Factsheet-201608.pdf}
  
\bibitem{kabbage}
  Mary Jane Credeur.
  \emph{How Kabbage Crowdsources Credit Scores}.
  Bloomberg, 2011.
  \url{https://www.bloomberg.com/news/articles/2011-09-15/how-kabbage-crowdsources-credit-scores}
  
\bibitem{lending-times}
  Geoerge Popescu.
  \emph{Lenddo - The Google of Lending Algorithms}.
  Lending Times, 2016.
  \url{http://lending-times.com/2016/02/29/lenddo-the-google-of-lending-algorithms/}
  
\bibitem{insurance-quotes}
  \emph{Your Social Media Could Affect Your Insurance Rates}.
  Insurance Quotes, 2015.
  \url{http://www.insurancequotes.org/auto/your-social-media-could-affect-your-insurance-rates/}
  
\bibitem{social-hiring}
  Chad Brooks.
  \emph{Social Screening: What Hiring Managers Look for On Social Media}.
  Business News Daily, 2016.
  \url{http://www.businessnewsdaily.com/2377-social-media-hiring.html}
  
\bibitem{insurance-further}
  Jathan Sadowski.
  \emph{Alarmed by Admiral's data grab? Wait until insurers can see the contents of your fridge}.
  The Guardian, 2016.
  \url{https://www.theguardian.com/technology/2016/nov/02/admiral-facebook-data-insurers-internet-of-things}
  
\bibitem{fico}
  Adrianne Jeffries.
  \emph{FICO may start including your Facebook presence in your credit score}.
  The Verge, 2014.
  \url{http://www.theverge.com/2014/1/9/5292568/fico-may-start-including-your-facebook-presence-in-your-credit-score}
  
\bibitem{stats}
  \emph{Social Networking Statistics}.
  Statistic Brain, 2016.
  \url{http://www.statisticbrain.com/social-networking-statistics/}
    
\end{thebibliography}

\end{document}
